% 最小示例文档 - 测试排版效果
\documentclass[12pt,oneside]{ctexbook}
\usepackage{style/custom}
\usepackage{pdfpages}

\title{操作系统项目工程文档\\[0.5em]\large 基于自制内核}
\author{作者姓名}
\date{\today}

\begin{document}

% ==================== 封面 ====================
\includepdf[pages=1]{figures/封皮.pdf}



% ==================== 目录 ====================
\pagenumbering{roman}
\tableofcontents
\clearpage
\pagenumbering{arabic}

% ==================== 第一章 ====================
\chapter{系统架构概述}

本章将介绍操作系统的整体架构设计,包括内核模块划分、启动流程以及核心数据结构。操作系统是计算机系统中最重要的软件之一,它负责管理硬件资源、提供用户接口以及运行应用程序。

\section{内核模块划分}

内核是操作系统的核心部分,负责直接与硬件交互。我们的操作系统采用模块化设计,将内核功能划分为以下几个主要模块。每个模块都有明确的职责和接口,便于维护和扩展。

\subsection{进程管理模块}

进程管理模块负责进程的创建、调度、同步和销毁。主要包括以下功能:

\begin{itemize}
    \item 进程控制块(PCB)管理
    \item 进程调度算法实现
    \item 进程间通信机制
    \item 信号处理
\end{itemize}

进程是操作系统资源分配的基本单位,每个进程都有独立的地址空间和执行状态。

\begin{figure}[H]
    \centering
    \includegraphics[width=0.6\textwidth]{figures/test.png}
    \caption{第一章图片A}
    \label{fig:ch1-a}
\end{figure}

\subsection{内存管理模块}

内存管理模块实现虚拟内存机制,包括页表管理、物理内存分配和页面置换算法。系统采用二级页表结构,支持4KB页面大小。

\begin{notebox}
    内存分配时需要注意对齐要求,否则可能导致性能下降或硬件异常。
\end{notebox}

\subsection{文件系统模块}

文件系统模块提供对磁盘数据的抽象访问接口。我们实现了一个类Unix的简单文件系统,支持inode结构、目录层次和基本的文件操作。

\begin{figure}[H]
    \centering
    \includegraphics[width=0.6\textwidth]{figures/test.png}
    \caption{第一章图片B - 文件系统示意}
    \label{fig:ch1-b}
\end{figure}

\section{启动流程分析}

系统启动分为多个阶段,从BIOS自检到内核初始化完成。以下是启动流程的详细说明。

\subsection{引导加载阶段}

BIOS完成自检后,将控制权交给引导扇区(MBR)。引导程序负责加载内核到内存并跳转执行。

\subsection{内核初始化阶段}

内核初始化包括:中断描述符表(IDT)设置、内存管理初始化、进程调度器初始化等。

% ==================== 第二章 ====================
\chapter{进程与线程}

本章详细介绍进程和线程的实现机制,包括数据结构设计和调度策略。

\section{进程控制块设计}

进程控制块(PCB)是描述进程状态的核心数据结构。表\ref{tab:pcb}列出了PCB的主要字段。

\begin{table}[htbp]
    \centering
    \begin{threeparttable}
        \caption{进程控制块主要字段}
        \label{tab:pcb}
        \begin{tabularx}{\textwidth}{XXX}
            \toprule
            字段名 & 类型 & 说明 \\
            \midrule
            pid & uint32\_t & 进程标识符\tnote{1} \\
            status & enum & 进程状态\tnote{2} \\
            priority & uint8\_t & 进程优先级 \\
            pgdir & uint32\_t* & 页表指针 \\
            parent & PCB* & 父进程指针 \\
            \bottomrule
        \end{tabularx}
        \begin{tablenotes}
            \footnotesize
            \item[1] 进程标识符在系统范围内唯一,由内核分配。
            \item[2] 状态包括:READY、RUNNING、BLOCKED、ZOMBIE。
        \end{tablenotes}
    \end{threeparttable}
\end{table}

\begin{figure}[H]
    \centering
    \includegraphics[width=0.8\textwidth]{figures/test.png}
    \caption{测试图片}
    \label{fig:test}
\end{figure}

\subsection{进程状态转换}

进程在其生命周期中会经历多种状态:就绪、运行、阻塞和终止。状态转换由调度器和系统调用触发。

\subsection{进程优先级管理}

系统支持多级优先级调度,优先级范围为0-31,数值越小优先级越高。

\section{线程实现}

线程是CPU调度的基本单位,多个线程可以共享同一进程的地址空间。

\subsection{内核线程}

内核线程运行在内核态,共享内核地址空间,用于执行系统任务。

\subsection{用户线程}

用户线程运行在用户态,通过系统调用请求内核服务。

% ==================== 第三章 ====================
\chapter{代码高亮预览}

本章展示 VS 主题的代码高亮效果,代码综合了常见的 C++ 语法元素。

\begin{figure}[H]
    \centering
    \includegraphics[width=0.7\textwidth]{figures/test.png}
    \caption{第三章图片 - 代码结构}
    \label{fig:ch3-code}
\end{figure}

\section{综合示例}

\begin{minted}[style=vs,linenos,breaklines]{c++}
// 预处理指令 (Preprocessor)
#include <iostream>
#include <vector>
#include <string>
#define MAX_SIZE 1024
#define DEBUG_MODE

// 命名空间 (Namespace)
namespace kernel {

// 枚举类型 (Enum)
enum class status : uint8_t {
    ready   = 0,
    running = 1,
    blocked = 2
};

// 类定义 (Class)
class thread {
public:
    // 类型别名 (Type Alias)
    using pid_t = uint32_t;
    
    // 构造函数
    explicit thread(const char* name, int priority = 31)
        : name_(name), priority_(priority), pid_(next_pid_++) {}
    
    // 成员函数 (Method)
    auto run() -> void {
        // 字符串 (String)
        const char* msg = "Hello World";
        std::string str = "Thread started";
        
        // 数字 (Number)
        int decimal = 42;
        double pi = 3.14159;
        uint32_t hex = 0xDEADBEEF;
        size_t binary = 0b11010010;
        
        // 运算符 (Operator)
        int result = decimal + 10 * 2 - 5;
        bool flag = (result > 0) && (priority_ <= 31);
        
        // 控制流 (Control Flow)
        if (flag) {
            for (size_t i = 0; i < MAX_SIZE; ++i) {
                while (status_ != status::ready) {
                    // 等待
                }
            }
        } else if (priority_ == 0) {
            return;
        }
        
        // 标准库调用
        std::cout << msg << std::endl;
        std::vector<int> vec = {1, 2, 3, 4, 5};
    }
    
    // 模板函数 (Template)
    template<typename T>
    auto print(const char* fmt, T&& arg) -> uint32_t {
        char buf[256]{};
        auto len = std::format_to(buf, fmt, std::forward<T>(arg));
        return static_cast<uint32_t>(len);
    }

private:
    const char* name_;
    int priority_;
    pid_t pid_;
    status status_ = status::ready;
    static inline pid_t next_pid_ = 1;
};

// 结构体 (Struct)
struct bitmap {
    uint8_t* bits;
    size_t sz;
    
    auto test(size_t bi) -> bool {
        return bits[bi / 8] >> (bi % 8) & 1;
    }
    
    auto set(size_t bi, bool value) -> void {
        if (value) {
            bits[bi / 8] |= 1 << (bi % 8);
        } else {
            bits[bi / 8] &= ~(1 << (bi % 8));
        }
    }
};

} // namespace kernel

// 主函数
int main() {
    auto t = kernel::thread{"init", 0};
    t.run();
    return 0;
}
\end{minted}

\end{document}
