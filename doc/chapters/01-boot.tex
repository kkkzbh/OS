\chapter{引导程序}

本章围绕仓库中 \mintinline{text}{boot/} 目录的实现展开,梳理从主引导记录(MBR)到 Loader 的完整流程:磁盘布局、实模式初始化、LBA 读盘、内存探测、进入保护模式以及为内核准备分页环境。重点解释高半区(0xC0000000 以上)执行所需的指令与数据布局。

\section{磁盘布局与常量约定}

引导阶段的内存与磁盘布局通过 \mintinline{text}{boot/include/boot.inc} 统一声明,涵盖 Loader 与内核的加载地址、分页用到的物理页框等常量:

\begin{table}[h]
    \centering
    \begin{tabular}{lll}
        \toprule
        标识符 & 含义 & 数值/扇区号 \\
        \midrule
        \mintinline{text}{LOADER_BASE_ADDR} & Loader 复制到内存的基址 & 0x900 (实模式可用) \\
        \mintinline{text}{LOADER_START_SECTOR} & Loader 在磁盘的起始 LBA 扇区 & 0x2 \\
        \mintinline{text}{KERNEL_BIN_BASE_ADDR} & 保护模式下临时存放 \mintinline{text}{kernel.bin} 的内存地址 & 0x70000 \\
        \mintinline{text}{KERNEL_START_SECTOR} & 内核镜像的磁盘起始扇区 & 0x9 \\
        \mintinline{text}{PAGE_DIR_TABLE_POS} & 建立分页结构使用的物理地址 & 0x0010\,0000 (1MB) \\
        \mintinline{text}{KERNEL_ENTRY_POINT} & 最终跳转到的内核虚拟地址入口 & 0xC0001500 \\
        \bottomrule
    \end{tabular}
    \caption{引导阶段关键常量,节选自 \mintinline{text}{boot/include/boot.inc}}
\end{table}

这些约定决定了后续章节中的地址推导:MBR 负责把 Loader 复制到 0x900,Loader 则在保护模式下把内核镜像搬运至 0x70000,并在分页开启后跳到 0xC0001500。



\section{复位向量与 BIOS 启动模型}

CPU 上电后从 \mintinline{text}{CS:IP=0xF000:0xFFF0}(物理地址 0xFFFF0)取指,执行 BIOS 固化代码完成 POST、自检与启动设备选择,随后把选中磁盘的 LBA0 扇区读入物理地址 0x7C00 并跳转。此时处理器处于 16 位实模式(分段寻址 \mintinline{text}{phys = 16*seg + off}),中断向量表位于 0:0。

\begin{notebox}[title=实模式小抄]
\begin{itemize}
  \item 段寄存器(CS/DS/ES/SS/FS/GS)均以 16 字节为粒度参与寻址:\mintinline{text}{phys = seg << 4 + off}。
  \item BIOS 提供常用中断:\mintinline{text}{int 0x10}(显示)、\mintinline{text}{int 0x13}(磁盘)、\mintinline{text}{int 0x15}(内存服务/E820)。
\end{itemize}
\end{notebox}

\section{MBR 布局速览}

主引导记录(MBR)占据磁盘第 0 个扇区(512B),布局如下:

\begin{longtable}{lll}
\toprule
偏移 & 大小 & 含义 \\
\midrule
0x000 & 446B & 启动代码(本项目:读取 Loader 至 0x900 并跳转) \\
0x1BE & 64B  & 分区表(4×16B,未在本项目解析) \\
0x1FE & 2B   & 启动签名 0x55AA \\
\bottomrule
\end{longtable}

本项目不依赖分区表,直接用常量 \mintinline{text}{LOADER_START_SECTOR=0x2} 指向 Loader 的起始扇区。

\section{执行流程概览}

在展开 MBR 细节之前,先给出从“上电”到“进入内核”的鸟瞰流程。仓库采用固定磁盘布局(MBR 位于 LBA0,Loader 从 LBA2 开始,内核从 LBA9 起),并在高半区(0xC0000000 以上)运行内核。

\begin{qsbox}[title=从上电到内核(Quick Overview)]
\begin{enumerate}
  \item 上电与 POST:BIOS 初始化最小硬件与中断向量表。
  \item 加载 MBR:BIOS 将 LBA0 读到 0x7C00 并跳转执行。
  \item MBR(16 位):统一段寄存器、显示启动标识,按 \mintinline{text}{LOADER_START_SECTOR}(默认 0x2)读取 Loader 至 0x900,跳转到 0x900:0x300。
  \item Loader(实模式):调用 \mintinline{text}{int 0x15}(优先 E820)探测物理内存,记录可用内存范围。
  \item 切换到保护模式:开启 A20,装载 GDT,设置 \mintinline{text}{CR0.PE=1},通过远跳转进入 32 位环境。
  \item 读取内核镜像:将 \mintinline{text}{kernel.bin} 自 \mintinline{text}{KERNEL_START_SECTOR}(默认 0x9)读入 0x70000 临时缓冲区。
  \item 解析并搬运:遍历 ELF Program Header,将各段复制到目标物理地址,为高半区执行做准备。
  \item 建立分页:在 1MB 处创建页目录与页表,使 PDE[0] 与 PDE[768] 指向同一张页表(低端同址映射 + 高半区镜像),最后一项指向自身以便递归映射。
  \item 开启分页与栈调整:写 \mintinline{text}{CR3},置位 \mintinline{text}{CR0.PG},把栈与数据段切换到高半区。
  \item 跳入内核:跳转到 \mintinline{text}{KERNEL_ENTRY_POINT}(0xC0001500),进入 C/C++ 内核初始化流程。
\end{enumerate}
\end{qsbox}

为便于查阅,表 \ref{tab:boot-stages} 汇总了各阶段的执行模式、入口与核心职责。

\begin{table}[h]
  \centering
  \caption{引导阶段总览}
  \label{tab:boot-stages}
  {\small
  \setlength{\tabcolsep}{6pt}
  \begin{tabularx}{\linewidth}{lll>{\raggedright\arraybackslash}X}
    \toprule
    阶段 & 模式 & 入口/地址 & 核心职责 \\
    \midrule
    BIOS & 实模式 & 加载到 0x7C00 & 读 LBA0 至 0x7C00,跳转 \\
    MBR  & 实模式 & 0x7C00 & 读 Loader 至 0x900,跳转 \\
    Loader(实→保) & 16→32 位 & 0x900:0x300→保护模式 & 内存探测、GDT、A20、读内核、建页表、开分页 \\
    Kernel & 高半区 & 0xC0001500 & 早期初始化(中断、内存管理等) \\
    \bottomrule
  \end{tabularx}}
\end{table}


\section{MBR:实模式初始化与 Loader 载入}

MBR 在 BIOS 把扇区装载到 0x7C00 后开始执行。本节将其拆分为更小的步骤逐一说明。

\subsection*{统一段寄存器与栈}
把 DS/ES/SS/FS 统一到 CS,并把栈指针设到 0x7C00,确保后续访问和调用环境稳定:

\codefilerange{nasm}{../boot/mbr.asm}{5}{12}

\subsection*{绑定 GS 指向文本显存}
将 GS 指向 0xB800 文本模式显存,便于直接写字符用于早期可视化调试:

\codefilerange{nasm}{../boot/mbr.asm}{13}{14}

\subsection*{清屏}
调用 BIOS 0x10(上卷屏幕)快速清空 25×80 文本区:

\codefilerange{nasm}{../boot/mbr.asm}{16}{22}

\subsection*{显示启动标识}
在显存首部写入“KKKZBH”,用于确认 MBR 已接管执行权:

\codefilerange{nasm}{../boot/mbr.asm}{23}{41}

\subsection*{读取 Loader 并跳转}
利用常量 \mintinline{text}{LOADER_START_SECTOR} 与 \mintinline{text}{LOADER_BASE_ADDR},读入四个扇区到 0x900,并跳到 0x900:0x300:

\codefilerange{nasm}{../boot/mbr.asm}{43}{49}

至此,控制权转交给功能更丰富的 Loader。

\section{16 位 LBA 读盘例程}

\mintinline{text}{rd_disk_m_16} 封装了 ATA 主通道的 PIO 读取流程,可分四步理解:

\subsection*{入口与寄存器备份}
将调用者传入的扇区号(EAX)与个数(CX)备份,便于后续拆分 LBA 地址:

\codefilerange{nasm}{../boot/mbr.asm}{50}{55}

\subsection*{编程扇区计数与 28 位 LBA}
按 0x1F2→0x1F6 顺序写入扇区数与 LBA[27:0] 四段:

\codefilerange{nasm}{../boot/mbr.asm}{57}{79}

\subsection*{下发命令并等待就绪}
向 0x1F7 写 0x20 触发 READ SECTORS,轮询 BUSY/DRQ 位至就绪:

\codefilerange{nasm}{../boot/mbr.asm}{80}{89}

\subsection*{PIO 数据传输循环}
一次读一个字(2B)共 256×扇区数次,搬入目标缓冲区:

\codefilerange{nasm}{../boot/mbr.asm}{91}{101}

上述流程在 Loader 的 32 位版本(\mintinline{text}{rd_disk_m_32})中同样适用,仅寄存器宽度与调用约定不同。

\section{Loader:BIOS 内存探测}

Loader 起始于 0xC00。源文件在真正进入 \codeinline{loader_start} 之前,先布置好 Loader 依赖的一组静态数据:包含空描述符、代码段、数据段与显存段的 GDT,三个段选择子(\codeinline{SELECTOR_CODE}/\codeinline{SELECTOR_DATA}/\codeinline{SELECTOR_VIDEO}),用于汇总结果的 \codeinline{total_mem_bytes},以及 E820 ARDS 缓冲区 \codeinline{ards_buf} 和其条目计数 \codeinline{ards_nr}。紧随其后的 \codeinline{gdt_ptr} 则打包了 GDT 的起始地址与长度,供稍后 \codeinline{lgdt} 指令加载:

\codefilerange{nasm}{../boot/loader.asm}{3}{42}

完成这些准备工作后,从 \codeinline{loader_start} 开始,Loader 通过 BIOS \mintinline{text}{int 0x15} 的多种子功能探测可用物理内存:优先使用 E820 SMAP 接口枚举所有 ARDS,若失败则依次回退至 E801 和 \codeinline{AH=0x88},最终把得到的物理内存上界统一写入 \codeinline{total_mem_bytes}:

\codefilerange{nasm}{../boot/loader.asm}{43}{114}

这段代码首先在 E820 正常可用的情况下循环调用该接口,把每条 ARDS(Address Range Descriptor)写入 \codeinline{ards_buf},累加 \codeinline{ards_nr},再按“基址 + 长度”的最高值选出最大的可用内存块并记录其末尾地址;若 E820 不可用,则退化到更老的 E801 或 AH=0x88,根据 BIOS 返回的 KB 计数换算得到 1MB 以上的物理内存容量,同样写入 \codeinline{total_mem_bytes},供内核阶段的物理页分配器使用。

\section{构建 GDT 并切换到保护模式}

内存探测完成后,Loader 需要完成三个同步动作:开启 A20 线、加载 GDT、设置 CR0 的 PE 位。片段如下:

\codefilerange{nasm}{../boot/loader.asm}{117}{139}

顺序上先通过端口 0x92 打开 A20 线,避免地址 1MB 回绕;再把 GDT 的指针装进 GDTR;最后把 \codeinline{CR0.PE} 置 1 并用一次远跳转刷新流水线,正式进入 32 位环境。

值得注意的实现细节:

\begin{itemize}
    \item GDT 包含空描述符、代码段、数据段与显存段,并预留了额外的 60 个槽位,方便内核阶段继续扩展。
    \item 通过写入 0x92 端口打开 A20 线,确保物理地址 1MB 以上的访问不会回绕。
    \item 使用 \mintinline{text}{lgdt [gdt_ptr]} 装载 GDT,再把 \mintinline{text}{cr0} 的最低位(PE)设为 1,最后以远跳转刷新流水线进入 32 位环境。
\end{itemize}

此后代码区段采用 \mintinline{text}{bits 32} 指令,全部运行在保护模式下。

\section{搬运内核与建立分页}

进入 32 位后,Loader 读取内核、解析 ELF,并建立分页。

\subsection*{读取内核到临时缓冲区}
两段连续读取覆盖 375 个扇区:

\codefilerange{nasm}{../boot/loader.asm}{150}{163}

此处调用 32 位版本的 PIO 读盘例程,把 \codeinline{kernel.bin} 搬到 0x70000 的临时缓冲区,便于后面按 Program Header 解析并重定位各段。

\subsection*{分页与高半区映射的切换步骤}
执行顺序是:构建页表(\mintinline{text}{setup_page})→ 迁移并重装 GDT → 写 CR3 → 置位 PG → 重装 GDT(高半区基址)→ 跳往内核入口:

\codefilerange{nasm}{../boot/loader.asm}{164}{188}

从实模式切到高半区的关键在于“低端映射 + 高半区镜像 + 自引用 PDE”三件套:这样既能在低地址完成初始化,又能无缝切换到 0xC0000000 以上执行,同时还能用递归地址方便地计算 PTE/PDE 的虚拟地址。

\subsection*{setup\_page:页目录与页表的构造}
完整函数如下,包含:PDE[0] 与 PDE[768] 指向同一页表、为低端 1MB 建立 PTE、预留内核页表目录项并设置“自引用 PDE”。

\codefilerange{nasm}{../boot/loader.asm}{244}{322}

这里用 0x0010\,0000 作为页目录与页表的物理基址:先填好第 0 项和第 768 项指向同一张页表,再把第 1023 项指回页目录自身。完成后把页目录基址写入 CR3、置位 CR0.PG,CPU 的地址翻译即刻切换到分页模式。

\begin{table}[h]
  \centering
  \caption{\mintinline{text}{setup_page} 构建后的关键映射}
  \label{tab:setup-page-mapping}
  {\small
  \setlength{\tabcolsep}{6pt}
  \begin{tabularx}{\linewidth}{>{\ttfamily}X >{\ttfamily}X >{\raggedright\arraybackslash}X}
    \toprule
    虚拟地址范围 & 对应物理范围/页框 & 用途 \\
    \midrule
    0x00000000--0x000FFFFF & 0x00000000--0x000FFFFF & \codeinline{PDE[0]} 身份映射 1MiB,Loader/设备仍可访问低端内存 \\
    0xC0000000--0xC00FFFFF & 0x00000000--0x000FFFFF & \codeinline{PDE[768]} 指向同一页表,提供内核高半区镜像 \\
    0xFFC00000--0xFFC00FFF & 0x00101000--0x00101FFF & 递归 PDE 映射下的第一个页表(管理低端 1MiB 与高半区镜像) \\
    0xFFFFF000--0xFFFFFFFF & 0x00100000--0x00100FFF & 页目录本体,可直接读写 \codeinline{PDE} 以增删映射 \\
    \bottomrule
  \end{tabularx}}
\end{table}

递归页目录的窗口也意味着 \codeinline{0xFFC00000 + n * 0x1000} 能映射到第 \codeinline{n} 张页表——例如 \codeinline{0xFFF00000} 正好是 \codeinline{PDE[768]} 的页表虚拟地址。这一技巧让分页结构在启用后依旧可由内核直接维护。

\subsection*{kernel\_init:按 Program Header 搬运段}
该函数遍历 Program Header,把每个段从 \mintinline{text}{p_offset} 拷贝到目标物理位置:

\codefilerange{nasm}{../boot/loader.asm}{198}{225}

处理过程中注意对齐:目标地址按页对齐以便后续映射,代码/数据段逐字搬运到正确位置;最后把栈与段寄存器切到高半区,并跳到内核入口完成接力。

最后,Loader 将 ESP 调整至高半区(0xC009F000),并跳转至 \mintinline{text}{KERNEL_ENTRY_POINT}(0xC0001500)。至此分页、栈与段表均就位,为后续内核初始化打下基础。

\subsection*{kkkzbh:被跳转的内核入口}
Loader 跳转的虚拟地址正好对应内核的 C 入口函数 \mintinline{text}{kkkzbh()}。\codefilerange{c}{../kernel/main.c}{3}{16} 展示了它的主体:输出问候语、打印一个样例十六进制值,随后调用 \codeinline{start()},该函数才会逐步启用异常、设备与内核子系统。也就是说,Bochs/GDB 里看到的“跳转到 0xC0001500”实际落在这个 C 函数上。

为了保证链接结果与 Loader 的跳转吻合,下面两段 CMake 设定展示了“生成内核可执行文件”以及“施加链接器选项”的关键部分:

\codefilerange{cmake}{../kernel/CMakeLists.txt}{156}{161}

\codefilerange{cmake}{../kernel/CMakeLists.txt}{183}{189}

其中第二段显式添加了 \codeinline{-Wl,-Ttext,0xC0001500} 与 \codeinline{-Wl,-e,kkkzbh}。前者把可执行文件的装载基址固定在 0xC0001500(高半区起点 + ELF header 偏移),后者把入口符号绑定到 \codeinline{kkkzbh},因此链接器会把 \codeinline{e_entry} 写成 0xC0001500,并在 ELF header 中声明“程序启动从 \codeinline{kkkzbh} 开始”。这一组合让 Loader、内核与工具链三方对入口地址的认知完全一致。
